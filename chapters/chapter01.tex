% Main chapter title
\chapter{Heading on Level 0 (chapter)}\label{chapter1} 

\lipsum[1]

%----------------------------------------------------------------------------------------
%	Heading on Level 1 (section)
%----------------------------------------------------------------------------------------

\section{Heading on Level 1 (section)}

\lipsum[2]

%-----------------------------------
%	Heading on Level 2 (subsection)
%-----------------------------------

\subsection{Heading on Level 2 (subsection)}

\lipsum[3]

%-----------------------------------
%	Heading on Level 3 (subsubsection)
%-----------------------------------

\subsubsection{Heading on Level 3 (subsubsection)}

\lipsum[4]

%-----------------------------------
%	Heading on Level 4 (paragraph)
%-----------------------------------

\paragraph{Heading on Level 4 (paragraph)}

\lipsum[5]

%----------------------------------------------------------------------------------------
%	Figures, Tables and Listings
%----------------------------------------------------------------------------------------

\section{Cite, Footnote, Figure, Table and Listings}

Lorem ipsum~\cite{loremlipsum} dolor sit amet, consectetuer adipiscing elit. Ut purus elit, vestibulumut, placerat ac, adipiscing vitae, felis. Curabitur dictum gravida mauris. 

\subsection{Figure}

The figure\ref{fig:appleblossom1} is a framed image of apples to be.

\begin{figure}[ht!]
    \frame{\includegraphics[width=0.7\linewidth]{images/lnu_apple_blossom.jpg}}
    \caption{Apple blossom.}\label{fig:appleblossom1}
\end{figure}

\subsection{Table}

The table\ref{table:simpletable} is a simple one.

\begin{table}[ht!]
    \begin{tabular}{ |c|c|c| } % chktex -2 (supress user regular expression warning)
        \hline                 % chktex -2 
        Col1 & Col2 & Col3 \\
        \hline                 % chktex -2 
        cell1 & cell2 & cell3 \\ 
        cell4 & cell5 & cell6 \\ 
        cell7 & cell8 & cell9 \\ 
        \hline                 % chktex -2 
    \end{tabular}
    \caption{A simple table.}\label{table:simpletable}
\end{table}

\subsection{Listings}

The listing\ref{lst:pythonhello} is a simple Python example.

\begin{lstlisting}[caption=Python example, label=lst:pythonhello, language=python]
# This is a comment
firstVariable = 'Hello World'
print(firstVariable.upper())
\end{lstlisting}

The listing\ref{lst:jshello} is a simple JavaScript\footnote{No official syntax highlighting is available for the JavaScript language.} example.

\begin{lstlisting}[caption=JavaScript example, label=lst:jshello]
// This is a comment
let firstVariable = 'Hello World'
console.log(firstVariable.toUpperCase())
\end{lstlisting}